% ================================================================================================= %

\documentclass[12pt,A4]{article}

%\usepackage{}

% ================================================================================================= %

\title
{
   Application of Formal Framework to Spiking Neural P System Variants: A Proposal
}

\author
{
   Ren Tristan A. de la Cruz
}

\date
{
   \today
}

% ================================================================================================= %

\begin{document}

\maketitle

% ================================================================================================= %

\section{Introduction}

\emph{Membrane computing} is a field of computer science that studies models of computation known as
\emph{P systems}. \emph{P systems} refer to a family of models of computation that are inspired by 
different biological processes. P systems use biological concepts like cells, cell membranes, neurons, 
tissues, etc. Rules (computation operations) in P systems are inspired by biological processes like
chemical reactions in cells, ion transport between regions divided by membranes, membrane creation,
division and dissolution, spiking of neurons, neurogenesis, and synaptogenesis.  

There is no single formal definition for the term `P system' but there are certain characteristics
that apply to most P systems. Most P systems use multiset of symbols as computing elements. One can
think of these symbols as molecules or ions. The multiset of symbols are compartmentalized into 
regions and these regions are defined by membranes that enclose them. i.e. Region 1 is the space
enclosed by membrane 1. This is the reason why the field is known as `membrane' computing. Regions
can be connected to each other. For example, in cell-like P system, membranes in the cell can be
nested. If there are two membranes in a cell, membrane $0$ and membrane $1$, and membrane $1$ is 
inside membrane $0$ then the region enclosed by membrane $0$ that is outside mebrane $1$ is 
`connected' to the region enclosed by membrane $1$. In tissue-like P system, membranes represent 
cells and a cell enclosed one region. Cells (and hence regions) are connected to each other via 
channels. Tissue P systems form networks of cells. In general, the \emph{configuration} of a P 
system refers to the network of cells/membranes with each cell/membrane containing a multiset of 
objects (the multiset can be empty). In some P systems, a cell/membrane has an internal state/status
that is different from its content (multiset of symbols). For example, in some cell-like P system, 
a membrane has a state known as \emph{polarity} or \emph{charge} which can either be negative, 
positive, or neutral.

The diversity of P systems primarily comes from diverse types of rules used by the systems.

\cite{freund-2007-ff-stat}
\cite{freund-2013-ff-dyn}
\cite{verlan-2014-ff-use}
\cite{verlan-2020-ff-snp}

hello world

% ================================================================================================= %

\section{Proposal}

% ================================================================================================= %


\bibliographystyle{plain}
\bibliography{cs-290-proposal}

% ================================================================================================= %

\end{document}

% ================================================================================================= %
