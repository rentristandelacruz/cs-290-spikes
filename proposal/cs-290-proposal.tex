% ================================================================================================= %

\documentclass[12pt,A4]{article}

\usepackage{geometry} % Added for margin

\geometry{top=25mm,
bottom=25mm,
left=30mm,
right=30mm,}


% ================================================================================================= %

\title
{
   Application of Formal Framework to Spiking Neural P System Variants: A Proposal
}

\author
{
   Ren Tristan A. de la Cruz
}

\date
{
   \today
}

% ================================================================================================= %

\begin{document}

\maketitle

% ================================================================================================= %

\section{Introduction}

\emph{Membrane computing} is a field of computer science that studies models of computation known as
\emph{P systems}. \emph{P systems} refer to a family of models of computation that are inspired by 
different biological processes. P systems use biological concepts like cells, cell membranes, neurons, 
tissues, etc. Rules (computating operations) in P systems are inspired by biological processes like
chemical reactions in cells, ion transport between regions divided by membranes, membrane creation,
division and dissolution, spiking of neurons, neurogenesis, and synaptogenesis. P systems are 
parallel and distributed models of computation. This means a P system can apply multiple rules at
the same time and these rules maybe applied in different parts/components of the P system. 

There is no single formal definition for the term `P system' but there are certain characteristics
that apply to most P systems. Most P systems use multiset of symbols as computing elements. One can
think of these symbols as molecules or ions. The multiset of symbols are compartmentalized into 
regions and these regions are defined by membranes that enclose them. i.e. Region 1 is the space
enclosed by membrane 1. This is the reason why the field is known as `membrane' computing. Regions
can be connected to each other. For example, in cell-like P system, membranes in the cell can be
nested. If there are two membranes in a cell, membrane $0$ and membrane $1$, and membrane $1$ is 
inside membrane $0$ then the region enclosed by membrane $0$ that is outside mebrane $1$ is 
`connected' to the region enclosed by membrane $1$. In tissue-like P system, membranes represent 
cells and a cell enclosed one region. Cells (and hence regions) are connected to each other via 
channels. Tissue P systems form networks of cells. In general, the \emph{configuration} of a P 
system refers to the network of cells/membranes with each cell/membrane containing a multiset of 
objects (the multiset can be empty). In some P systems, a cell/membrane has an internal state/status
that is different from its content (multiset of symbols). For example, in some cell-like P system, 
a membrane has a state known as \emph{polarity} or \emph{charge} which can either be negative, 
positive, or neutral.

At the time of writing, there are probably hundreds of P system models. One reason for the
diversity of P systems is the diversity of the types of rules used by the systems. There are many
different types of rules and a P system model can use more than one type.  A rule is an 
operation that transforms the configuration of a P system. One can look at a rule as having two
components: (1) a set of conditions the configuration of a P system has to meet for the rule to
be eligible for application, and (2) a set of actions (transformations) the rule will apply to
the P system's configuration. Different types of rules have different sets of conditions and
different actions since they are inspired by different biological processes. For example, one common
type of rule is called an \emph{evolution rule}. An \emph{evolution rule} is associated with a
region and it consumes a multiset of objects in that region then produces another multiset of 
objects. Evolution rules are inspired by chemical reaction inside the membranes of a cell. Another
example of a rule type is the \emph{communication rule}. In cell-like P systems, a communication
rule is associated with a membrane and it facilitates a transfer or exchange of multisets of objects
between the region inside the membrane and the region outside. Communication rules are inspired by
the ion transfers between regions via ion channels on cell membranes. There are also types of rules
that transform the network of cells/membranes itself and not just the contents of the 
cells/membranes. For example, there are rules that can create and delete channels between cells. 
There are also rules that can create and delete membrane/cells. These rules are inspired by processes
like cell/membrane division, membrane dissolution, neuron creation, synapse creation and deletion, 
etc.

P systems are parallel models so it is possible for the systems to apply different rules at
the same time and they can also apply a rule multiple times. For example, evolution rules are 
inspired by chemical reactions and it is possible for different chemical reactions to occur multiple
times inside cell membranes. P systems have a set of criteria that specifies which combination of
eligible rules are valid. This set of criteria is called the \emph{derivation mode}. Aside from the
types of rules used by P systems, different P systems can also use different derivation modes.
Derivation modes can describe the type of parallelism the P system is working on. For example, the
derivation mode known as \emph{maximal parallelism} only allows combinations of eligible rule 
instances that are \emph{maximal}. In a \emph{maximal} combination of rule instances, one can not
add any instance of any eligible rules. i.e The maximal combination of rules will consume enough 
symbols such that the remaining multiset of symbols does not allow any more rules to be applied.
In a P system working in maximally parallel mode, any rule combinations of eligible rules that are
not maximal are not allowed. Another common derivation mode is the \emph{minimally parallel} mode. 
In P systems working in minimally parallel mode, cells/membranes with eligible rules can still apply
rules in parallel but each cell/membrane can only apply a single eligible rule once. Derivation
modes can also describe rule priorities. For example, in a P system with two types of rules, type
$A$ and type $B$, one can specify that in the model if any type $A$ rule in a membrane is eligible 
then no (eligible) type $B$ rules can be applied. In the situation where there is an eligible type
$A$ rule in a membrane, any rule combination that includes an eligible type $B$ rule in the same 
membrane is not valid. 

When one combines the different types of rules and the different derivation modes, one can produce
a significant number of diverse P system models. In order to help make sense of the different P 
system models a \emph{formal framework} was introduced in 2007 by Freund et al. The idea behind the
formal framework (for P systems) is to have a set of general concepts and constructs that can be 
used when analysing most (if not all) types of P systems. The first version of the formal framework
can be found in \cite{freund-2007-ff-stat} and it is a framework for static tissue-like P system.
Similar to most P systems, a configuration in the framework is a network of cells that contain 
multisets of symbols. The framework has a single type of rule called an \emph{interaction rule}.
An \emph{interaction rule} is a rule type that generalizes most rule types used by cell-like and 
tissue-like P systems. For example, most variants of communication rules and evolution rules are
special cases of the more general interaction rule. Aside from having different forms, different 
types of rules can also have significantly different syntax. By using the interaction rule form,
P system rules can be written to have the same general form and a common syntax. The interaction 
rule form makes comparing rules of different P systems easier. For example, two rule types from
two different P systems may superficially appear different but when they are written as interaction
rules it appears that their eligibility criteria and their actions are actually very similar. In the
formal framework, effect of an interaction rule when applied to configuration is formally defined. 
This is not necessarily the case for other P systems and their rules. In some P system models,
the effect of the rules on the configuration is described in a less formal manner (describe in plain
English) which is prone to different interpretation. If one is to write a rule type from one P 
system model as an interaction rule, one has to formally define the meaning of the rule. The 
framework helps one clarify the meaning of the rule.

Aside from a general form of a rule, the framework in \cite{freund-2007-ff-stat} also lists common
derivation modes used by most P systems. The meaning of those different derivation modes are 
formally defined. In some P systems, derivation modes are less formally defined which again can lead 
to multiple conflicting interpretations. The framework lists other possible derivation modes that
are rarely or not yet used by existing P system models.

The formal framework from \cite{freund-2007-ff-stat} is limited to static P system. One can not 
represent a P system rule that changes the network of cells itself and not simply the content of the
cells using the interaction rule in \cite{freund-2007-ff-stat}. i.e. The rules that can create and 
delete cells and connections between cells can not be written as interaction rules. The second
version of the formal framework \cite{freund-2013-ff-dyn} has a much more general form of a rule.
This rule form can be used to write rules that change the structure of the network of cells. The 
rule form is much more complicated that the form of the interaction rules in 
\cite{freund-2007-ff-stat}. A third version of the formal framework can be found in 
\cite{verlan-2020-ff-snp}. This version is actually an extension of the first version. The main
differences between the first and the third version are: (1) the interaction rule in the third
version allows multiset pattern checking and (2) the third version framework introduces the input
and output constructs. In the first version of the framework, the criteria for eligibility of
interaction rules are combination of presence of certain multisets in specified regions and absence
of certain multisets in specified regions. In the third version, the criteria for eligibility of an
interaction rule rely on \emph{control languages} (the `pattern'). For example, if certain multisets 
in specified regions are in the control languages of those regions (the multisets `fits the pattern'
for those regions), then the rule is eligible. Control languages and the input and output
constructs are needed to represent P systems known as \emph{spiking neural P systems} (SN P systems). 
SN P systems use a type of rule that checks for patterns. Such rules need the concept of a control 
language in order for them to be represented in the framework. SN P systems are also usually used as
transducers so it makes sense for the new framework to have input and output constructs.

The formal frameworks can be used to understand the functioning of some P system variants. This is
done by translating concepts of a P system model to concepts in the framework.
Since the concepts in the framework are formally defined, the process of translating
P system concepts to the language of the framework forces one to clarify and formalize
those concepts that may have been vaguely described in the P system model. This process
is particularly helpful when analysing P system models with vague semantics. The process highlights
the concepts (e.g. rule semantics) that may have different interpretations. The formal frameworks
can also be used as a common language for comparing different P systems. By translating different
P systems as models in the formal framework, the formal framework models for the different P systems
will have the same general form and will use the same syntax. This makes the comparison of those
models a significantly easier task. The formal framework can also be use to extend P system models
with new features. Since the constructs in the formal framework can be seen as generalized versions
of constructs in most P system models (i.e. formal framework's interaction rule is a generalized
P system rule), one can add features available in the formal framework constructs as new features in
the P system model. These of the formal framework can be found in \cite{verlan-2014-ff-use}.

% ================================================================================================= %

\section{Proposal}

We want to do an exploratory study of the different SN P system variants using the formal 
frameworks.

SN P systems and similar variants are neural-like P systems. The rules of these models are inspired 
by the mechanism of a \emph{spiking neuron}. In most SN P system variants, there is only a one type 
of symbol called a \emph{spike}. The configuration of an SN P system is a network of neurons and 
each neuron contains a multiset of spikes. The main type of rule used by SN P system variants is
called a \emph{spiking rule}. A spiking rule looks at number of spikes in a certain neuron and if 
the number of spikes fits the pattern (usually described by a regular expression) specified in the 
rule then the neuron can send a spike to adjacent neurons in the network. The interaction rule in 
third version of the formal framework \cite{verlan-2020-ff-snp} was specifically extended to have 
control languages in order to be able to represent the spiking rule as interaction rules.

There are many SN P system variants that have features that are not available in the original SN P
system model. For example, there is a variant with two types of objects known as the \emph{spike}
and the \emph{anti-spike}. The variant is called SN P system \emph{with anti-spikes}. The spiking
rule of this variant needs to take into account the new type of object which means both its syntax
and semantics are different from the original spiking rule. It also works in a slightly different 
derivation mode. There is a variant called SN P system \emph{with astrocytes}. It has a new 
construct called \emph{astrocyte}. An astrocyte can be connected to different channels and it
monitors the spikes travelling in those channels. It is a threshold mechanism that allows 
travelling spikes to propagate if the total number of spikes travelling in the channels being 
monitored by the astrocyte passes a certain threshold. There is a variant called SN P system
\emph{with polarity}. Similar to some cell-like P systems, in SN P systems with polarity, neurons
have charges that can either be positive, negative or neutral. The spiking rules in this model not
only check the number of spikes in the neuron but also the polarity of the neuron. There are also
SN P system variants that are exactly like the original SN P system except for the fact that they
use a different derivation modes. For example, \emph{sequential} SN P systems are SN P systems that
only allows a single rule to be applied at a time. Another example is the \emph{asynchronous} SN P
system. This variant allows any combination of eligible rules (the combination does not have to be
minimally parallel).

In \cite{verlan-2020-ff-snp}, after defining and describing the constructs of the framework, the
authors translated the SN P system model to a model of the framework. They wrote both the spiking
rule and forgetting rule of the SN P system as interaction rules. They mentioned that the SN P
system works in minimally parallel mode which has already been characterized as the derivation mode
$min_1$ in \cite{freund-2007-ff-stat}. Only original SN P system model (without delay) and SN P 
systems with extended rules (also without delay) have been fully translated as models of the 
framework. Some features of other SN P variants were translated by Verlan et al. in 
\cite{verlan-2020-ff-snp} and in some conference presentations (Conference on Membrane Computing)
but not the full models.

Our intention for the study is to look at different SN P system variants and fully translate them
to models of the framework. We want to fully translate at least two SN P variants. After translating
that variants, we can them compare the framework models with each other. This can give us insights
on how similar or how different those SN P system variants are to each other.

% ================================================================================================= %

\bibliographystyle{plain}
\bibliography{cs-290-proposal}

% ================================================================================================= %

\end{document}

% ================================================================================================= %
