% ================================================================================================= %
\documentclass[12pt,A4]{article}

\title
{
   CS 290 Paper Commentaries
}

\author
{
   Ren Trista A. de la Cruz
}

\date
{
   \today
}

% ================================================================================================= %

\begin{document}

\maketitle

% ================================================================================================= %

\section*{Computing with Spikes}

\emph{Computing with Spikes} \cite{maass-2002-comp-spike} gives a quick overview of the idea behind 
computing models based on \emph{spiking neurons} and the (then) current research that the author
Wolfgang Maass and his colleagues were conducting. A significant portion of the paper is dedicated
to describing the mechanisms of a spiking neuron.

The field that studies spiking neuron models (or spiking neural networks) is in the intersection of 
the field of neuroscience and the field of theoretical computer science. Unlike abstract models of 
computation like Turing  machines and counter machines in theoretical computer science, spiking 
neuron models are significantly more complex since they are used to model and study the
computational process of the nervous system. Spiking neuron models should be abstract enough for 
them to be used as models of computation that can solve abstract problems (in theoretical computer
science) but they should be detailed enough so that they can be used to explain phenomena in the
nervous system (in neuroscience).

Artificial neural networks are computing models that are `inspired' biological neural network but the 
activation mechanism (using real functions like sigmoid and trigonometric functions) of the neuron 
in these models does not represent how activation occurs in a biological neuron. The activation 
mechanism in a spiking neuron model more closely represents the activation mechanism of a biological 
neuron. Aside from the difference in activation mechanism, the organization of an artificial neural 
network is also different to the organization of a spiking neural network.  

Maass et al.'s research involves determining the computational function of \emph{neural 
microcircuits}. In a spiking neuron model, these microcircuits are model using spiking neurons. The 
research asks how exactly does a spiking neuron work. The research involves studying actual
biological neurons in order to observe how they produce and process spikes. Other aspects of their 
research involve the study of how the spiking neurons are organized (into networks), how memory are
stored in the networks, and how learning is done by the networks. 

A large part of the paper describes the spiking neuron and gives a general idea behind the spiking 
mechanism. The neuron has three main parts: the \emph{soma}, the \emph{dendritic tree}, and the
\emph{axonal tree}. The soma is the body of the neuron that produces the signal called 
\emph{spikes}. The dendritic tree is the `input' region of the neuron where it receives signals from 
other neurons. The axonal tree is the `output' region of the neuron where it sends out signals to 
other neurons. A part of the axonal tree (output region) of one neuron can be `connected' to a part 
of the dendritic tree (input region) of another neuron. This `connection' or interface between a 
neuron's axonal tree and another neuron's dendritic tree is called a \emph{synapse}.  

A neuron has a \emph{membrane potential}, a voltage value based on the difference between the charge
inside and the charge outside the membrane of the neuron. A neuron has a resting membrane potential 
which is about $-70$ millivolts. The `spike' in a spiking neuron model refers to an \emph{action 
potential} in a neuron. An action potential is a sudden increase (around $40$ millivolts) and then a 
sudden decrease of the neuron's membrane potential (happens in less than $3$ milliseconds). The term 
`spike' refers to that even of the membrane potential spiking. 

When a neuron receives a certain combination of inputs (spikes from other neurons received by the 
dendritic tree), it will produce a spike. There is threshold mechanism in a neuron. If the 
`combination' of input spikes passes a certain threshold, the neuron will spike. The amplitude of 
the spike does not change for a neuron. The input spikes can dictate if the neuron will spike or not 
but not how `large' the spike is. A series of input spikes can, however, dictate the timing (i.e.
frequency) in which the neuron spikes.

When a neuron spikes, the spike travels along the axon then reaches the axon terminals. The axon
terminals can be connected to dendrites of other neurons. The synapse, that connects the axon 
terminal to dendrite of another neuron, is responsible from `processing' the spike and then passing 
a `processed' spike to the next neuron. The synapse has an internal state/configuration. This state
is affected by the spikes it receives. In a sense, this is a form of memory. This state affects how
the synapse processed incoming spikes.   

In the biological neuron, the spiking event is a result of a combination of electrochemical 
activities that involve the neuron membrane, voltage-gate ion channels, potassium and sodium ions, 
etc. The activities in a synapse involve neurotransmitters, neurotransmitter transporters and
receptors, etc. In the spiking neuron models, these electrochemical activities are abstracted and
simplified in order to have manageable models of computation but they are still somewhat faithful to 
their biological analogues (at least much more faithful compared to artificial neural network 
models). 

A spiking neuron can only produce a spike, a single type of signal. A neuron can not produce a 
`large' spike or `small' spike. All spikes produced are identical. Information in spiking neuron
models are encoded by spiking patterns (in time) produced by a neuron. This pattern in known as
a \emph{spike train}. For example, a spike can be represented by `$1$' and no spike represented by
`$0$'. A spiking neuron's activities (spiking or resting) can be represented by a string (the spike
train) over the binary alphabet $\{0,1\}$. This string is a pattern in time instead of being a
pattern in space. One can only observe the spike train by looking at the activities of a neuron for
a certain period of time.

In summary, a spiking neuron model is  a model of computation based on the spiking mechanism of a
biological neuron. It is much more complex than other models of computation because it also has the
role of modelling neural phenomena. Spiking neural network s are much more similar to biological 
neural network unlike artificial neural networks.

% ================================================================================================= %

\section*{Spiking Neural P Systems}

% ================================================================================================= %

\section*{}

% ================================================================================================= %

\bibliographystyle{plain}
\bibliography{cs-290-reviews}

\end{document}
