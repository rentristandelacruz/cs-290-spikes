% ================================================================================================= %
\documentclass[12pt,A4]{article}

\title
{
   CS 290 Paper Reviews
}

\author
{
   Ren Trista A. de la Cruz
}

\date
{
   \today
}

% ================================================================================================= %

\begin{document}

\maketitle

% ================================================================================================= %

\section*{Computing with Spikes}

\emph{Computing with Spikes} \cite{maass-2002-comp-spike} gives a quick overview of the idea behind 
computing models based on \emph{spiking neurons} and the (then) current research that the author
(Wolfgang Maass) and his colleagues were conducting.


The field that studies spiking neurons models is in the intersection of complexity theory and
computational learning theory.

Theoretical ideas regarding cognition and learning
organization of information processing in the human brain

Questions
Computational function of neural microcircuit
principles by which learning and memory are organized
organization principles by which millions of neural microcircuit can communicate and collaborate 
in our brains

temporal dynamics of a neuron
time instead of frequency
neuron, soma, dendritic trees, axonal trees, synapses,
neurotransmitter, channels, 
spike- sudden voltage increase, action potential 
excitatory post-synaptic potentials
inhibitory post-synaptic potentials

no synchronization, mixed digital and analog 
synapse as learning component


% ================================================================================================= %

\section*{Spiking Neural P Systems}

% ================================================================================================= %

\section*{}

% ================================================================================================= %

\bibliographystyle{plain}
\bibliography{cs-290-reviews}

\end{document}
