\documentclass[a4paper]{article}

\usepackage{geometry}            % for margins
\usepackage{amsthm}              % for theorems, definition, etc.
\usepackage{amssymb}             % for mathcal, mathbb, mathfrak
\usepackage{amsmath}             % for \text
\usepackage{mathrsfs}            % for \mathscr
\usepackage{multicol}            % for \multicols
\usepackage[hidelinks]{hyperref} % for clickable and hidden reference links

\geometry
{
   left   = 25mm,
   right  = 25mm,
   top    = 25mm,
   bottom = 25mm,
}
\theoremstyle{definition}
\newtheorem{definition}{Definition}

\newcommand{\ra}{\rightarrow}
\newcommand{\ts}{\text{ }}
\newcommand{\mn}{\text{-}}

\begin{document}


\title
{
   Spiking Neural P System Models\\ as Formal Framework Models
}


\author
{
   Ren Tristan A. de la Cruz
}


\maketitle

% ================================================================================================= %

\section{Background}

% ================================================================================================= %

\section{Preliminaries}

The following sets will be commonly used throughout the document: $\mathbb{N} = \{0,1,2,3,...\}$,
$\mathbb{N}_{\infty} = \mathbb{N} \cup \{\infty\}$, $[1..n] = \{1,...,n\}$, 
$2^{[1..n]}=\mathcal{P}([1..n])$ (power set of $[1..n]$).

Let $V$ be a set of symbols called an \emph{alphabet}. A \emph{string} or \emph{word} over $V$ is
a sequence of symbols from $V$. The \emph{empty string} $\epsilon$ is a string without symbols, an 
empty sequence. A \emph{string language} over $V$ is a set of strings over $V$. The set of all 
strings over $V$ is denoted by $V^*$.

A \emph{multiset} over $V$ is a function of the form $m: V \ra \mathbb{N}_{\infty}$ while a
\emph{finite multiset} over $V$ is a function of the form $m: V \ra \mathbb{N}$. If $m$ is a
multiset over $V$ and $a \in V$, $m(a)$ denotes the number of occurrences of symbol $a$ in multiset
$m$. If $V=\{v_1,...,v_k\}$ and $m$ is a finite multiset over $V$, $m$ can be represented by the
string ${v_1}^{m(v_1)}\cdots {v_k}^{m(v_k)}$. The size of a multiset $m$ over $V$ is
$|m| = \sum_{v\in V}m(v)$. An \emph{empty multiset} $\emptyset$ is any multiset of size $0$. A 
\emph{multiset language} over $V$ is a set of multisets over $V$. The set of all multisets over $V$ is denoted by
$V^{\circ}$.

Let $V=\{v_1,...,v_k\}$, $m$ be the multiset ${v_1}^{m(v_1)}\cdots{v_k}^{m(v_k)}$ and $n$ be the  
multiset ${v_1}^{n(v_1)}\cdots{v_k}^{n(v_k)}$. $m \subseteq n$ if and only if for all $v \in V$ 
$m(v) \leq n(v)$. $m+n$ is the multiset ${v_1}^{m(v_1)+n(v_1)}\cdots {v_k}^{m(v_k)+n(v_k)}$. If 
$m \subseteq n$, $n-m$ is the multiset ${v_1}^{n(v_1)-m(v_1)}\cdots {v_k}^{n(v_k)-m(v_k)}$

The set of $n$-vectors whose components are finite multisets over $V$ is denoted by
${V^{\circ}}^{n}$. Let $X=(x_1,...,x_n), Y=(y_1,...,y_n) \in {V^{\circ}}^n$. $X \subseteq Y$ if and
only if $x_i \subseteq y_i$ for $1 \leq i \leq n$. $X+Y = (x_1 + y_1,...,x_n + y_n)$. If 
$X \subseteq Y$, $Y-X = (y_1-x_1,...,y_n-x_n)$. Aside from denoting the empty multiset, $\emptyset$
will also denote a vector of empty multisets. i.e. $\emptyset = (\emptyset,...,\emptyset)$. If the
context is not clear, we will specify if $\emptyset$ means an empty multiset or a vector of empty
multisets.  

A \emph{family of languages} is a set of languages. It can either be a family of string languages or 
a family of multiset languages. The notations $\mathscr{F}$ and $\mathscr{F}^{\circ}$ will be used
for a family of string languages and a family of multiset languages, respectively. The notation
$\mathscr{F}(V)$ will be used for a family of string languages over alphabet $V$ while 
$\mathscr{F}(V)^{\circ}$ will be used for a family of multiset languages over alphabet $V$. For
example, $REG$ is the set regular string languages, $REG(V)$ is the set of regular string 
languages over $V$, $REG^{\circ}$ is the set of regular multiset languages, and $REG(V)^{\circ}$ 
is the set of regular multiset languages over $V$.

% ================================================================================================= %

\section{Formal Framework for Spiking Neural P Systems}

% ================================================================================================= %

\definition{}\label{def-conf} \textbf{[Configuration]} An $n$-degree \emph{configuration} 
$C = (u_1,...,u_n)$ over alphabet $V$ is an $n$-vector of multisets over $V$. A configuration $C$ is 
called a \emph{finite configuration} if all the components are finite multisets. 

A configuration is referred to as a \emph{full configuration} if we specifically want to specify 
that the configuration can contain infinite multisets.

% ================================================================================================= %

\definition{}\label{def-rule} \textbf{[Interaction Rule]} An $n$-degree \emph{interaction rule} over
alphabet $V$ is the construct $(X \ra Y; K)$ where $X=(x_1,...,x_n)$ and $Y=(y_1,...,y_n)$ 
are $n$-vectors of multisets over $V$ and $K = (k_1,...,k_n)$ is an $n$-vector of multiset
languages.

$${[x_1]}_1\cdots {[x_n]}_n \ra {[y_1]}_1\cdots{[y_n]}_n\ts ;\ts {[k_1]}_1\cdots{[k_n]}_n$$
The multiset languages $k_1,...,k_n$ are called \emph{control languages}.

$$(1,x_1)\cdots(n,x_n)\ra (1,y_1)\cdots(n,y_n)\ts;\ts (1,k_1)\cdots(n,k_n)$$

if $x_i$ or $y_i$ is $\emptyset$ the term $(i,x_i)$ and $(i,y_i)$ can be removed


if $x_i$ (or $y_i$) is empty, the term $[x_i]_i$ (or $[y_i]_i$) can be removed.

% ================================================================================================= %

\definition{}\label{def-elig} \textbf{[Rule Eligibility]} Let $C = (u_1,...,u_n)$ be an $n$-degree
configuration over $V$ and $r = ((x_1,...,x_n) \ra (y_1,...,y_n);$ $(k_1,...,k_n))$ be an $n$-degree
rule over $V$, rule $r$ is \emph{eligible} with respect to configuration $C$ if the following 
conditions hold: (1) for all $x_i$, $x_i \subseteq u_i$ and (2) for all $u_i$, $u_i \in k_i$.

% ================================================================================================= %

\definition{}\label{def-appc} \textbf{[Applicability of a Multiset of Rules]} Let $R'$ be a multiset 
of $n$-degree rules over $V$ and $C = (u_1,...,u_n)$ be an $n$-degree configuration over $V$, $R'$ 
is \emph{applicable} to configuration $C$ if the following conditions hold: (1) each rule 
$(X_j \ra Y_j;K_j) \in R'$ is eligible with respect to configuration $C$ and (2) 
$$X = \Bigg(\sum_{\substack{(X_j \ra Y_j; K_j) \in R'}} X_j\Bigg)\subseteq C.$$

% ================================================================================================= %

\definition{}\label{def-appl} \textbf{[Application of a Multiset of Rules]} If $R'$ is multiset of 
rules applicable to configuration $C = (u_1,...,u_n)$, \emph{applying} $R'$ to configuration $C$ 
means producing a new configuration $C'$ which is defined as: 
$$C' = Apply(R',C) =  C - \Bigg(\sum_{\substack{(X_j \ra Y_j; K_j) \in R'}} X_j\Bigg) +
\Bigg(\sum_{\substack{(X_j \ra Y_j; K_j) \in R'}} Y_j\Bigg).$$

The \emph{application function} $Apply(R',C)$

% ================================================================================================= %

\definition{}\label{def-nc} \textbf{[Network of Cells]} A $\mathscr{F}$-controlled \emph{network of 
cells} of degree $n$ is the construct $$\Pi = (n, V, W, c_{in}, c_{out}, R)$$ where

\begin{multicols}{2}
\begin{itemize}
\item $n$ is the number of cells; 
\item $V$ is a finite alphabet;
\item $W = (w_1,...,w_n)$ is the \emph{initial configuration} and $w_i \in V^{\circ}$ is the
      multiset associated with cell $i$.
\item $c_{in} \subseteq \{1,...,n\}$ is the set of \emph{input cells}.
\item $c_{out} \subseteq \{1,...,n\}$ is the set of \emph{output cells}.
\item $R$ is the set of interactive rules. 
\end{itemize}
\end{multicols}

$Applicable(\Pi,C)$ is the set of  multiset of rules (rules from $R$) applicable to configuration 
$C$.

% ================================================================================================= %

\definition{}\label{[def-nc2]}\textbf{[Network of Cells with Environment]} An 
$\mathscr{F}$-controlled \emph{network of cells with environment} of degree $n$ is the construct:

$$\Pi_{inf} = (\Pi,Inf) = ((n,V,W,c_{in},c_{out},R), Inf)$$

\begin{multicols}{2}
\begin{itemize}
\item The first component $\Pi_{inf}$ is as defined in Definition \ref{def-nc}.
\item $Inf = (inf_1,...,inf_n)$ where $inf_i \subseteq V$ is a \emph{set of symbols occurring 
infinitely often in cell i}.
\end{itemize}
\end{multicols}

% ================================================================================================= %

\definition{}\label{def-derv} \textbf{[Derivation Mode]} A \emph{derivation mode} $\delta$ is a
restriction of the set of applicable rules. For an $\mathscr{F}$-controlled network of cells $\Pi$
and configuration $C$, $Appl(\Pi, C, \delta) \subseteq Appl(\Pi, C)$ denotes the set of multisets of
rules in $\Pi$ applicable to configuration $C$ according to derivation mode $\delta$.

% ================================================================================================= %

\definition{}\label{def-nc3}\textbf{[Network of Cells Working in $\delta$ Derivation Mode]} An
$\mathscr{F}$-controlled $n$-degree \emph{network of cells working in $\delta$ derivation mode} is 
the construct: $$\Pi' = (\Pi,\delta) = ((n,V,W,c_{in},c_{out},R), \delta)$$
\begin{multicols}{2}
\begin{itemize}
\item The first component $\Pi_{inf}$ is as defined in Definition \ref{def-nc}.
\item $\delta$ is the derivation mode used.
\end{itemize}
\end{multicols}

$Applicable(\Pi, R', \delta)$

$NC(n,V,\mathscr{F},\delta)$ is the set of $n$-degree $\mathscr{F}$-controlled network of cells 
using alphabet $V$ and working in $\delta$ derivation mode.

% ================================================================================================= %

\definition{}\label{def-comp1}\textbf{[Computation of a Network of Cells]} A \emph{computation} of
a network of cell $\Pi' = ((n,V,W,c_{in},c_{out},R),\delta)$ is sequence a $C_0,C_1,C_2,...$ with 
the following properties:
\begin{itemize}
\item $C_0 = W$
\item $C_{i+1} = Apply(\Pi',R',C_i)$ where $R' \in Applicable(\Pi,C_i,\delta)$.
\end{itemize}

% ================================================================================================= %

\definition{}\label{def-input} \textbf{[Input Function]} An \emph{input function} for a system 
$\Pi'=((n,v,W,c_{in},c_{out},R),\delta)$, $\Pi' \in NC(n,V,\mathscr{F},\delta)$, is a function of
the form $Input(\Pi'): \mathbb{N} \ra {V^{\circ}}^{n}$ and fulfills that condition that for all
$i \notin c_{in}$ the $i$-th component of resulting input vector from ${V^{\circ}}^n$ is an empty
multiset.

% ================================================================================================= %

\definition{}\label{def-comp2} \textbf{[Computation of a Network of Cells]} 
\begin{itemize}
\item $C_0 = W + Input(\Pi')(0)$
\item $C_{i+1} = Apply(\Pi',C_i,R') + Input(\Pi')(i+1)$
where $R' \in Appl(\Pi, C_i, \delta)$.
\end{itemize}
% ================================================================================================= %

\definition{}\label{def-output} \textbf{[Output Function]}

% ================================================================================================= %

\section{Spiking Neural P System Models as Formal Framework Network of Cells} 


\subsection{Spiking Neural P System}

\definition{}\label{def-snp} \textbf{[Spiking Neural P System]} A \emph{spiking neural P sytem} of
degree $n$ the construct:

$$\Pi = (O, \sigma_1,...,\sigma_n,syn,i_o)$$

\begin{itemize}
\item $O = \{a\}$ is the singleton alphabet of the system. Symbol $a$ is called a \emph{spike}.
\item $\sigma_1,...,\sigma_n$ are neurons of the form
      \begin{itemize}
      \item $\sigma_i = (n_i, R_i)$ for $1 \leq i \leq n$. 
      \item $n_i \in \mathbb{N}$ is initial number of spikes in neuron $i$. 
      \item $R_i$ is a finite set of rules of the following two forms:
            \begin{itemize}
            \item \emph{Spiking Rule:} $E/a^c \ra a;d$ where $E$ is regular expression over $O$, 
                  $d \in \mathbb{N}$,$c \in \mathbb{N}\backslash \{0\}$.
            \item \emph{Forgetting Rule:} $a^s \ra \lambda$ where $s \in \mathbb{N}\backslash\{0\}$\
                  and $\{a^s\} \cap L^{\circ}(E) = \emptyset$ for all $E$ that are regular 
                  expressions of a spiking rules in the same neuron.
            \end{itemize}
      \end{itemize}
\item $syn \subseteq [1..n] \times [1..n]$ is the set of \emph {synapses} where $(i,i) \notin syn$ 
      for all $i \in [1..n]$.
\item $i_o \in [1..n]$ specifies the \emph{output neuron}.
\end{itemize}

\noindent \textbf{SNP System' Spiking Rule:}
$${[a^c]}_i \ra {[a]}_{j_1} \cdots [a]_{j_k};{[L^{\circ}(E)]}_i$$

\noindent \textbf{SNP System's Forgetting Rule}
$${[a^c]}_i \ra \emptyset;{[L^{\circ}(E)]}_i$$

\noindent \textbf{SNP Systems with  Extended Spiking Rule}
$${[a^c]}_i \ra {[a^p]}_{j_1} \cdots {[a^p]}_{j_k};{[L^{\circ}(E)]}_i$$

\noindent \textbf{SNP Systems with Weights}
$$[a^c]_i \ra [a^{w_{j_1}}]_{j_1} \cdots [a^{w_{j_k}}]_{j_k};[L^{\circ}(E)]_i$$

\noindent \textbf{SNP Systems with Extended Rules and  Weights}
$$[a^c]_i \ra [a^{p \cdot w_{j_1}}]_{j_1} \cdots [a^{p\cdot w_{j_k}}]_{j_k};[L^{\circ}(E)]_i$$

% ================================================================================================= %

\bibliographystyle{plain}
\bibliography{cs-290}

\end{document}
